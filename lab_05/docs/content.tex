\chapter{Задача <<Производство-потребление>>}

\section{Вывод программы}

На рисунках \ref{img:pc55} -- \ref{img:pc83} показан вывод программы, реализующей задачу <<Производство-потребление>> при разных значениях задержки потребителей и производителей. Примечание: для удобства \textbf{\color{forestgreen}производители} обозначены зеленым, в то время как \textbf{\color{red}потребители} --- красным; в скобках указано время последнего сна.

\img{165mm}{pc55}{<<Производство-потребление>> при задержках: потребители $\in [0, 5)$, производители $\in [0, 5)$}
\clearpage
\img{240mm}{pc38}{<<Производство-потребление>> при задержках: потребители $\in [0, 3)$, производители $\in [0, 8)$}
\clearpage
\img{240mm}{pc83}{<<Производство-потребление>> при задержках: потребители $\in [0, 8)$, производители $\in [0, 3)$}
\clearpage

\section{Листинги кода}

В листингах \ref{lst:buffer} -- \ref{lst:main1} представлены исходные коды программы, реализующей задачу <<Производство-потребление>>.

\begin{lstinputlisting}[
        caption={Реализация очереди, на основе циклического массива (буфера)},
        label={lst:buffer},
        linerange={1},
        style={cstyle}
    ]{../produce_consume/src/buffer.c}
\end{lstinputlisting}

\begin{lstinputlisting}[
        caption={Реализация ``производителей'' и ``потребителей''},
        label={lst:runners},
        linerange={1},
        style={cstyle}
    ]{../produce_consume/src/runners.c}
\end{lstinputlisting}

\begin{lstinputlisting}[
        caption={Файл с константными значениями},
        label={lst:constants1},
        linerange={1},
        style={cstyle}
    ]{../produce_consume/include/constants.h}
\end{lstinputlisting}

\begin{lstinputlisting}[
        caption={Главный файл программы},
        label={lst:main1},
        linerange={1},
        style={cstyle}
    ]{../produce_consume/src/main.c}
\end{lstinputlisting}

\chapter{Задача <<Читатели-Писатели>>}

\section{Вывод программы}

На рисунке \ref{img:rw} показан вывод программы, реализующей задачу <<Читатели-писатели>>. Примечание: для удобства \textbf{\color{forestgreen}писатели} обозначены зеленым, в то время как \textbf{\color{red}читатели} --- красным; в скобках указано время последнего сна.

\img{180mm}{rw}{<<Читатели-писатели>>}

\section{Листинги кода}

В листингах \ref{lst:io_obj} -- \ref{lst:main2} представлены исходные коды программы, реализующей задачу <<Производство-потребление>>.

\begin{lstinputlisting}[
        caption={Реализация ``читателей'' и ``писателей''},
        label={lst:io_obj},
        linerange={1},
        style={cstyle}
    ]{../read_write/src/io_objects.c}
\end{lstinputlisting}

\begin{lstinputlisting}[
        caption={Файл с константными значениями},
        label={lst:constants2},
        linerange={1},
        style={cstyle}
    ]{../read_write/include/constants.h}
\end{lstinputlisting}

\begin{lstinputlisting}[
        caption={Главный файл программы},
        label={lst:main2},
        linerange={1},
        style={cstyle}
    ]{../read_write/src/main.c}
\end{lstinputlisting}
