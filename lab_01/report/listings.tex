\section*{Листинг обработчика INT 8h}

\begin{lstlisting}[style={asm}]
;; вызов sub_1
020A:0746   E8 0070             call sub_1 ; (07B9)
;; Сохранение регистров ES, DS, AX, DX
020A:0749   06                  push es
020A:074A   1E                  push ds
020A:074B   50                  push ax
020A:074C   52                  push dx
;; DS = 0040
020A:074D   B8 0040             mov ax,40h
020A:0750   8E D8               mov ds,ax
;; AX = 0
020A:0752   33 C0               xor ax,ax ; Zero register
020A:0754   8E C0               mov es,ax
;; 0040:006Ch - адрес счетчикa таймера
020A:0756   FF 06 006C          inc word ptr ds:[6Ch] ; (0040:006C=5AEBh)
020A:075A   75 04               jnz loc_1 ; Jump if not zero
;; 0040:006Еh - старшие 2 байта счетчика таймера
020A:075C   FF 06 006E          inc word ptr ds:[6Eh] ; (0040:006E=2)
020A:0760               loc_1:
;; Проверка: 0040:006Eh == 18h (24) И 0040:006Ch == B0h (176)
;; Можно убедиться в том, что: 18h << 16 + B0h = 24 * 60 * 60 * freq, 
;; где freq - кол-во раз, которое вызывается таймер в секунду.
;; Таким образом из того, что условие выполняется, следует, что прошли сутки.
020A:0760   83 3E 006E 18       cmp word ptr ds:[6Eh],18h ; (0040:006E=2)
020A:0765   75 15 jne loc_2 ; Jump if not equal
020A:0767   81 3E 006C 00B0     cmp word ptr ds:[6Ch],0B0h ; (0040:006C=5AEBh)
020A:076D   75 0D jne loc_2 ; Jump if not equal
;; Зануление счетчика (старшего слова и младшего слова)
020A:076F   A3 006E             mov word ptr ds:[6Eh],ax ; (0040:006E=2)
020A:0772   A3 006C             mov word ptr ds:[6Ch],ax ; (0040:006C=5AEBh)
;; Прошло более 24 часов, занесение значения 1 в 0040:0070
020A:0775   C6 06 0070 01       mov byte ptr ds:[70h],1 ; (0040:0070=0)
;; AL = 8 (потому что ax до этого момента = 0)
020A:077A   0C 08               or al,8
020A:077C               loc_2:
020A:077C   50                  push ax
;; Декремент счетчика отключения моторчика
020A:077D   FE 0E 0040          dec byte ptr ds:[40h] ; (0040:0040=0F7h)
020A:0781   75 0B               jnz loc_3 ; Jump if not zero
;; Установка флага отключения моторчика дисковода (1-3 биты == 0)
020A:0783   80 26 003F F0       and byte ptr ds:[3Fh],0F0h ; (0040:003F=0)
;; 3 строчки - посылка команды отключения дисководу
020A:0788   B0 0C               mov al,0Ch
020A:078A   BA 03F2             mov dx,3F2h
020A:078D   EE                  out dx,al ; port 3F2h, dsk0 contrl output
020A:078E               loc_3:
020A:078E   58                  pop ax
;; Проверка 2 бита (PF)
020A:078F   F7 06 0314 0004     test word ptr ds:[314h],4 ; (0040:0314=3200h)
020A:0795   75 0C               jnz loc_4 ; Jump if not zero
;; Копирование младшего байта FLAGS в ah
020A:0797   9F                  lahf ; Load ah from flags
;; Смена мест: 
;; теперь в ax: 08XXh - где XX - младший байт FLAGS
020A:0798   86 E0               xchg ah,al
;; Кладем это на стек и вызываем прерывание
020A:079A   50                  push ax
;; Вызываем 1Сh через адрес в таблице векторов. До этого мы добавили в стек AX, в то время как
;; вызов int делает push флагов (то есть наш ax, описанный 6 строками выше будет как FLAGS в 1Ch)
020A:079B   26: FF 1E 0070      call dword ptr es:[70h] ; (0000:0070=6ADh)
020A:07A0   EB 03               jmp short loc_5 ; (07A5)
020A:07A2   90                  nop
020A:07A3               loc_4:
020A:07A3   CD 1C               int 1Ch ; Timer break (call each 18.2ms)
020A:07A5               loc_5:
020A:07A5   E8 0011             call sub_1 ; (07B9)
;; Сброс контроллера прерываний
; al = 20h, end of interrupt
020A:07A8   B0 20               mov al,20h ; ' '
020A:07AA   E6 20               out 20h,al ; port 20h, 8259-1 int command
;; Восстановление регистров
020A:07AC   5A                  pop dx
020A:07AD   58                  pop ax
020A:07AE   1F                  pop ds
020A:07AF   07                  pop es
020A:07B0   E9 FE99 jmp $-164h ; (020A:07B0h - 164h = 020A:064Ch)
;; ...  -164h
020A:064C   1E                  push ds
020A:064D   50                  push ax
;; ...
020A:06AA   58                  pop ax
020A:06AB   1F                  pop ds
020A:06AC   CF iret ; Interrupt return
\end{lstlisting}

\pagebreak
\section*{Листинг процедуры sub\_1.}
\begin{lstlisting}[style={asm}]
sub_1       proc                near
;; Сохранение регистров
020A:07B9   1E                  push ds
020A:07BA   50                  push ax
020A:07BB   B8 0040             mov ax,40h
020A:07BE   8E D8               mov ds,ax
;; Младший байт FLAGS в AH
020A:07C0   9F                  lahf ; Load ah from flags
;; Установлены ли старший бит IOPL или DF?
020A:07C1   F7 06 0314 2400     test word ptr ds:[314h],2400h ; (0040:0314=3200h)
020A:07C7   75 0C               jnz loc_7 ; Jump if not zero
;; сброс IF в 0040:0314h (зануление 9 бита)
020A:07C9   F0> 81 26 0314 FDFF lock and word ptr ds:[314h],0FDFFh ; (0040:0314=3200h)
020A:07D0               loc_6:
;; AH копируется в младший байт FLAGS
020A:07D0   9E                  sahf ; Store ah into flags
020A:07D1   58                  pop ax
020A:07D2   1F                  pop ds
020A:07D3   EB 03 jmp short loc_8 ; (07D8)
020A:07D5               loc_7:
;; Сброс IF
020A:07D5   FA                  cli ; Disable interrupts
020A:07D6   EB F8               jmp short loc_6 ; (07D0)
020A:07D8               loc_8:
020A:07D8   C3                   retn
sub_1       endp
\end{lstlisting}
